\chapter{Foreword and Syllabus}\label{Foreword-and-Syllabus}

\begin{quote}
\emph{``Human ingenuity cannot concoct a cipher which human ingenuity
cannot resolve.''} Edgar Allan Poe, 1841
\end{quote}

Cryptography - the art or science of ``secret writing'' - has been
around for several millenia, and for almost all of that time Edgar Allan
Poe's quote above held true. Indeed, the history of cryptography is
littered with the figurative corpses of cryptosystems believed secure
and then broken, and sometimes with the actual corpses of those who have
mistakenly placed their faith in these cryptosystems. Yet, something
changed in the last few decades. New cryptosystems have been found that
have not been broken despite being subjected to immense efforts
involving both human ingenuity and computational power on a scale that
completely dwarves the ``crypto breakers'' of Poe's time. Even more
amazingly, these cryptosystem are not only seemingly unbreakable, but
they also achieve this under much harsher conditions. Not only do
today's attackers have more computational power but they also have more
data to work with. In Poe's age, an attacker would be lucky if they got
access to more than a few ciphertexts with known plaintexts. These days
attackers might have massive amounts of data- terabytes or more - at
their disposal. In fact, with \emph{public key} encryption, an attacker
can generate as many ciphertexts as they wish.

These new types of cryptosystems, both more secure and more versatile,
have enabled many applications that in the past were not only impossible
but in fact \emph{unimaginable}. These include secure communication
without sharing a secret, electronic voting without a trusted authority,
anonymous digital cash, and many more. Cryptography now supplies crucial
infrastructure without which much of the modern ``communication
economy'' could not function.

This course is about the story of this cryptographic revolution.
However, beyond the cool applications and the crucial importance of
cryptography to our society, it contains also intellectual and
mathematical beauty. To understand these often paradoxical notions of
cryptography, you need to think differently, adapting the point of view
of an attacker, and (as we will see) sometimes adapting the points of
view of other hypothetical entities. More than anything, this course is
about this cryptographic way of thinking. It may not be immediately
applicable to protecting your credit card information or to building a
secure system, but learning a new way of thinking is its own reward.

~

\section{Syllabus}\label{Syllabus}

In this fast-paced course, I plan to start from the very basic notions
of cryptography and by the end of the term reach some of the exciting
advances that happened in the last few years such as the construction of
\emph{fully homomorphic encryption}, a notion that Brian Hayes called
``one of the most amazing magic tricks in all of computer science'', and
\emph{indistinguishability obfuscators} which are even more amazing. To
achieve this, our focus will be on \emph{ideas} rather than
\emph{implementations} and so we will present cryptographic notions in
their pedagogically simplest form-- the one that best illustrates the
underlying concepts-- rather than the one that is most efficient, widely
deployed, or conforms to Internet standards. We will discuss some
examples of practical systems and attacks, but only when these serve to
illustrate a conceptual point.

Depending on time, I plan to cover the following notions:

\begin{itemize}
\item
  Part I: Introduction

  \begin{enumerate}
  \def\labelenumi{\arabic{enumi}.}
  \item
    \textbf{How do we define security for encryption?} Arguably the most
    important step in breaking out of the ``build-break-tweak'' cycle
    that Poe's quote described has been the idea that we can have a
    \emph{mathematically precise definition} of security, rather than
    relying on fuzzy notions, that allow us only to determine with
    certainty that a system is \emph{broken} but never have a chance of
    \emph{proving} that a system is \emph{secure} .
  \item
    \textbf{Perfect security and its limitations:} Showing the
    possibility (and the limitations) of encryptions that are perfectly
    secure regardless of the attacker's computational resources.
  \item
    \textbf{Computational security:} Bypassing the above limitations by
    restricting to computationally efficient attackers. Proofs of
    security by reductions.
  \end{enumerate}
\item
  Part II: Private Key Cryptography

  \begin{enumerate}
  \def\labelenumi{\arabic{enumi}.}
  \item
    \textbf{Pseudorandom generators:} The basic building block of
    cryptography, which also provided a new twist on the age-old
    philosophical and scientific question of the nature of randomness.
  \item
    \textbf{Pseudorandom functions, permutations, block ciphers:} Block
    ciphers are the working horse of crypto.
  \item
    \textbf{Authentication and active attacks:} \emph{Authentication}
    turns out to be as crucial, if not more, to security than
    \emph{secrecy} and often a precondition to the latter. We'll talk
    about notions such as Message Authentication Codes and
    Chosen-Ciphertext-Attack secure encryption, as well as real-world
    examples why these notions are necessary.
  \item
    \textbf{Hash functions and the ``Random Oracle Model'':} Hash
    functions are used all over in crypto, including for verifying
    integrity, entropy distillation, and many other cases.
  \item
    \textbf{Building pseudorandom generators from one-way permutations
    (optional):} Justifying our ``axiom'' of pseudo-random generators by
    deriving it from a weaker assumption.
  \end{enumerate}
\item
  Part III: Public key encryption

  \begin{enumerate}
  \def\labelenumi{\arabic{enumi}.}
  \item
    \textbf{Public key cryptography and the obfuscation paradigm:} How
    did Diffie, Hellman, Merkle, Ellis even dare to \emph{imagine} the
    possibility of public key encryption?
  \item
    \textbf{Constructing public key encryption: Factoring, discrete log,
    and lattice based systems:} We'll discuss several variants for
    constructing public key systems, including those that are widely
    deployed such as RSA, Diffie-Hellman, and the elliptic curve
    variants, as well as some variants of \emph{lattice based
    cryptosystems} that have the advantage of not being broken by
    quantum computers, as well as being more versatile. The former is
    the reason why the NSA has advised people to transition to
    lattice-based cryptosystems in the not too far future.
  \item
    \textbf{Signature schemes:} These are the public key versions of
    authentication though interestingly are easier to construct in some
    sense than the latter.
  \item
    \textbf{Active attacks for encryption:} Chosen ciphertext attacks
    for public key encryption.
  \end{enumerate}
\item
  Part IV: Advanced notions

  \begin{enumerate}
  \def\labelenumi{\arabic{enumi}.}
  \item
    \textbf{Fully homomorphic encryption:} Computing on encrypted data.
  \item
    \textbf{Multiparty secure computation:} An amazing construction that
    enables applications such as playing poker over the net without
    trusting the server, privacy preserving data mining, electronic
    auctions without a trusted auctioneer, electronic elections without
    a trusted central authority.
  \item
    \textbf{Zero knowledge proofs:} Prove a statement without revealing
    the reason to \emph{why} its true.
  \item
    \textbf{Quantum computing and cryptography:} Shor's algorithm to
    break RSA and friends. Quantum key distribution. On ``quantum
    resistant'' cryptography.
  \item
    \textbf{Indistinguishability obfuscation:} Construction of
    indistinguishability obfuscators, the potential ``master tool'' for
    crypto.
  \item
    \textbf{Practical protocols:} Techniques for constructing practical
    protocols for particular tasks as opposed to general (and often
    inefficient) feasibility proofs.
  \item
    \textbf{Cryptocurrencies:} Hash chains and Merkle trees, proofs of
    work, achieving consensus on a ledger via ``majority of cycles'',
    smart contracts, achieving anonymity via zero knowledge proofs.
  \end{enumerate}
\end{itemize}

\subsection{Prerequisites}\label{Prerequisites}

The main prerequisite is the ability to read, write (and even enjoy!)
mathematical proofs. In addition, familiarity with algorithms, basic
probability theory and basic linear algebra will be helpful. We'll only
use fairly basic concepts from all these areas: e.g.~Oh-notation-
e.g.~\(O(n)\) running time- from algorithms, notions such as events,
random variables, expectation, from probability theory, and notions such
as matrices, vectors, and eigenvectors. Mathematically mature students
should be able to pick up the needed notions on their own. See the
``mathematical background'' handout for more details.

No programming knowledge is needed. If you're interested in the course
but are not sure if you have sufficient background, or you have any
other questions, please don't hesitate to contact me.

\section{Why is cryptography hard?}\label{Why-is-cryptography-hard}

Cryptography is a hard topic. Over the course of history, many brilliant
people have stumbled in it, and did not realize subtle attacks on their
ciphers. Even today it is frustratingly easy to get crypto wrong, and
often system security is compromised because developers used crypto
schemes in the wrong, or at least suboptimal, way. Why is this topic
(and this course) so hard? Some of the reasons include:

\begin{itemize}
\item
  To argue about the security of a cryptographic scheme, you have to
  think like an attacker. This requires a very different way of thinking
  than what we are used to when developing algorithms or systems, and
  arguing that they perform well.
\item
  To get robust assurances of security you need to argue about \emph{all
  possible attacks} . The only way I know to analyze this infinite set
  is via \emph{mathematical proofs} . Moreover, these types of
  mathematical proofs tend to be rather different than the ones most
  mathematicians typically work with. Because the proof itself needs to
  take the viewpoint of the attacker, these often tend to be proofs by
  contradiction and involve several twists of logic that take some
  getting used to.
\item
  As we'll see in this course, even \emph{defining} security is a highly
  non trivial task. Security definitions often get subtle and require
  quite a lot of creativity. For example, the way we model in general a
  statement such as ``An attacker Eve does not get more information from
  observing a system above what she knew a-priori'' is that we posit a
  ``hypothetical alter ego'' of Eve called Lilith who knows everything
  Eve knew a-priori but does not get to observe the actual interaction
  in the system. We then want to prove that anything that Eve learned
  could also have been learned by Lilith. If this sounds confusing, it
  is. But it is also fascinating, and leads to ways to argue
  mathematically about \emph{knowledge} as well as beautiful
  generalizations of the notion of encryption and protecting
  communication into schemes for protecting \emph{computation} .
\end{itemize}

If cryptography is so hard, is it really worth studying? After all,
given this subtlety, a single course in cryptography is no guarantee of
using (let alone inventing) crypto correctly. In my view, regardless of
its immense and growing practical importance, cryptography is worth
studying for its \emph{intellectual} content. There are many areas of
science where we achieve goals once considered to be science fiction.
But cryptography is an area where current achievements are so fantastic
that in the thousands of years of secret writing people did not even
dare \emph{imagine} them. Moreover, cryptography may be hard because it
forces you to think differently, but it is also rewarding because it
teaches you to think differently. And once you pass this initial hurdle,
and develop a ``cryptographer's mind'', you might find that this point
of view is useful in areas that seem to have nothing to do with crypto.
