\chapter{More obfuscation, exotic
encryptions}\label{More-obfuscation-exotic-encryp}

Fully homomorphic encryption is an extremely powerful notion, but it
does not allow us to obtain fine control over the access to information.
With the public key you can do all sorts of computation on the encrypted
data, but you still do not learn it, while with the private key you
learn everything. But in many situations we want \emph{fine grained
access control}: some people should get access to some of the
information for some of the time. This makes the ``all or nothing''
nature of traditional encryptions problematic. While one could still
implement such access control by interacting with the holder(s) of the
secret key, this is not always possible.

The most general notion of an encryption scheme allowing fine control is
known as \emph{functional encryption}, as was described in the previous
lecture. This can be viewed as an object dual to Fully Homomorphic
Encryption, and incomparable to it. For every function \(f\), we can
construct an \emph{\(f\)-restricted decryption key} \(d_f\) that allows
recovery of \(f(m)\) from an encryption of \(m\) but not anything else.

In this lecture we will focus on a weaker notion known as \emph{identity
based encryption (IBE)}. Unlike the case of full fledged functional
encryption, there are fairly efficient constructions known for IBE.

\section{Slower, weaker, less securer}\label{Slower-weaker-less-securer}

In a sense, functional encryption or IBE is all about selective
\emph{leaking} of information. That is, in some sense we want to modify
an encryption scheme so that it actually is ``less secure'' in some very
precise sense, so that it would be possible to learn something about the
plaintext even without knowing the (full) decryption key.

There is actually a history of cryptographic technique meant to support
such operations. Perhaps the ``mother'' of all such ``quasi encryption''
schemes is the modular exponentiation operation \(x \mapsto g^x\) for
some discrete group \(\mathbb{G}\). The map \(x \mapsto g^x\) is not
exactly an encryption of \(x\)- for one thing, we don't know how to
decrypt it. Also, as a deterministic map, it cannot be semantically
secure. Nevertheless, if \(x\) is random, or even high entropy, in
groups such as cyclic subgroup of a multiplicative group modulo some
prime, we don't know how to recover \(x\) from \(g^x\). However, given
\(g^{x_1},\ldots,g^{x_k}\) and \(a_1,\ldots,a_k\) we can find out if
\(\sum a_i x_i = 0\), and this can be quite useful in many applications.

More generally, even in the private key setting, people have studied
encryption schemes such as

\begin{itemize}
\item
  \textbf{Deterministic encryption} : an encryption scheme that maps
  \(x\) to \(E(x)\) in a deterministic way. This cannot be semantically
  secure in general but can be good enough if the message \(x\) has high
  enough entropy or doesn't repeat and allows to check if two
  encryptions encrypt the same object. (We can also do this by
  publishing a hash of \(x\) under some secret salt.)
\item
  \textbf{Order preserving encryption}: is an encryption scheme mapping
  numbers in some range \(\{1,\ldots, N \}\) to ciphertexts so that
  given \(E(x)\) and \(E(y)\) one can efficiently compare whether
  \(x<y\). This is quite problematic for security. For example, given
  \(poly(t)\) random such encryptions you can more or less know where
  they lie in the interval up to \((1 \pm 1/t)\) multiplicative factor..
\item
  \textbf{Searchable encryption}: is a generalization of deterministic
  encryption that allows some more sophisticated searchers (such as not
  only exact match).
\end{itemize}

Some of these constructions can be quite efficient. In particular the
system \href{https://css.csail.mit.edu/cryptdb/}{CryptDB} developed by
Popa et al uses these kinds of encryptions to automatically turn a SQL
database into one that works on encrypted data and still supports the
required queries. However, the issue of how dangerous the ``leakage''
can be is somewhat subtle. See this
\href{http://cryptoonline.com/wp-content/uploads/2013/03/edb.pdf}{paper}
and
\href{https://outsourcedbits.org/2015/09/07/attacking-encrypted-database-systems/}{blog
post} claiming weaknesses in practical use cases for CryptDB, as well as
this \href{https://css.csail.mit.edu/cryptdb/response.html}{response} by
the CryptDB authors.

While the constructions of IBE and functional encryption often use maps
such as \(x \mapsto g^x\) as subroutines, they offer a stronger control
over the leakage in the sense that, in the absence of publishing a
(restricted) decryption key, we always get at least CPA security.

\section{How to get IBE from pairing based
assumptions.}\label{How-to-get-IBE-from-pairing-ba}

The standard exponentiation mapping \(x \mapsto g^x\) allows us to
compute \emph{linear functions in the exponent}. That is, given any
linear map \(L\) of the form \(L(x_1,\ldots,x_k) = \sum a_i x_i\), we
can efficiently compute the map
\(g^{x_1},\ldots, g^{x_k} \mapsto g^{L(x_1,\ldots,x_k)}\). But can we do
more? In particular, can we compute \emph{quadratic} functions? This is
an issue, as even computing the map \(g^x,g^y \mapsto g^{xy}\) is
exactly the Diffie Hellman problem that is considered hard in many of
the groups we are interested in.

\emph{Pairing based cryptography} begins with the observation that in
some elliptic curve groups we can use a map based on the so called Weil
or Tate pairings. The idea is that we have an efficiently computable
isomorphism from a group \(\mathbb{G}_1\) to a group \(\mathbb{G}_2\)
mapping \(g\) to \(\hat{g}\) such that we can efficiently map the
elements \(g^x\) and \(g^y\) to the element
\(\varphi(g^x,g^y)=\hat{g}^{xy}\). This in particular means that given
\(g^{x_1},\ldots,g^{x_k}\) we can compute
\(\hat{g}^{Q(x_1,\ldots,x_k)}\) for every quadratic \(Q\). Note that we
cannot repeat this to compute, say, degree \(4\) functions in the
exponent, since we don't know how to invert the map \(\varphi\).

The \textbf{Pairing Diffie Hellman Assumption} is that we can find two
such groups \(\mathbb{G}_1,\mathbb{G}_2\) and generator \(g\) for
\(\mathbb{G}\) such that there is no efficient algorithm \(A\) that on
input \(g^a,g^b,g^c\) (for random
\(a,b,c \in \{0,\ldots,|\mathbb{G}|-1\}\)) computes \(\hat{g}^{abc}\).
That is, while we can compute a quadratic in the exponent, we can't
compute a cubic.

We now show an IBE construction due to Boneh and Franklin\footnote{The
  construction we show was first published in the CRYPTO 2001
  conference. The Weil and Tate pairings were used before for
  cryptographic attacks, but were used for a positive cryptographic
  result by Antoine Joux in his 2000 paper getting a three-party Diffie
  Hellman protocol and then Boneh and Franklin used this to obtain an
  identity based encryption scheme, answering an open question of
  Shamir. At approximately the same time as these papers, Sakai, Ohgishi
  and Kasahara presented a paper in the SCIS 2000 conference in Japan
  showing an identity-based key exchange protocol from pairing. Also
  \href{https://en.wikipedia.org/wiki/Clifford_Cocks}{Clifford Cocks}
  (who as we mentioned above in the 1970's invented the RSA scheme at
  GCHQ before R,S, and A did), also came up in 2001 with a different
  identity-based encryption scheme using the quadratic residuosity
  assumption.} how we can obtain from the pairing diffie hellman
assumption an identity based encryption:

\begin{itemize}
\item
  \textbf{Master key generation:} We generate
  \(\mathbb{G}_1,\mathbb{G}_2,g\) as above, choose \(a\) at random in
  \(\{0,\ldots,|\mathbb{G}|-1\}\). The master private key is \(a\) and
  the master public key is \(\mathbb{G}_1,\mathbb{G}_2,g,h=g^a\). We let
  \(H:\{0,1\}^*\rightarrow\mathbb{G}_1\) and
  \(H':\mathbb{G}_2\mapsto\{0,1\}^\ell\) be two hash functions modeled
  as random oracles.
\item
  \textbf{Key distribution:} Given an arbitrary string
  \(id\in\{0,1\}^*\), we generate the decryption key corresponding to
  \(id\), as \(d_{id} = H(id)^a\).
\item
  \textbf{Encryption:} To encrypt a message \(m\in\{0,1\}^\ell\) given
  the public paramters and some id \(id\), we choose
  \(c\in \{0,\ldots,|\mathbb{G}|-1\}\), and output
  \(g^c,H'(id\|\varphi(h,H(id))^c) \oplus m\)
\item
  \textbf{Decryption:} Given the secret key \(d_{id}\) and a ciphertext
  \(h',y\), we output \(H'(id\|\varphi(d_{id},h'))\oplus x\)
\end{itemize}

\paragraph{Correctness:} We claim that \(D_{d_{id}}(E_{id}(m))=m\).
Indeed, write \(h_{id}=H(id)\) and let \(b=\log_g h_{id}\). Then an
encryption of \(m\) has the form
\(h'=g^c, H'(id\|\varphi(g^a,h_{id})^c) \oplus m\), and so the second
term is equal to \(H'(id\|\hat{g}^{abc}) \oplus m\). However, since
\(d_{id}= h_{id}^a = g^{ab}\), we get that
\(\varphi(h',d_{id})=\hat{g}^{abc}\) and hence decryption will recover
the message. QED

\paragraph{Security:} To prove security we need to first present a
\emph{definition} of IBE security. The definition allows the adversary
to request keys corresponding to arbitrary identities, as long as it
does not ask for keys corresponding to the target identity it wants to
attack. There are several variants, including CCA type of security
definitions, but we stick to a simple one here:

\paragraph{Definition:} An IBE scheme is said to be CPA secure if every
efficient adversary Eve wins the following game with probability at most
\(1/2+ negl(n)\):

\begin{itemize}
\tightlist
\item
  The keys are generated and Eve gets the master public key.
\item
  For \(i=1,\ldots,T=poly(n)\), Eve chooses an identity
  \(id_i \in \{0,1\}^*\) and gets the key \(d_{id}\).
\item
  Eve chooses an identity \(id^* \not\in \{id_1,\ldots,id_T\}\) and two
  messages \(m_0,m_1\).
\item
  We choose \(b\leftarrow_R\{0,1\}\) and Eve gets the encryption of
  \(m_b\) with respect to the identity \(id^*\).
\item
  Eve outputs \(b'\) and \emph{wins} if \(b'=b\).
\end{itemize}

\paragraph{Theorem:} If the pairing Diffie Hellman assumption holds and
\(H,H'\) are random oracles, then the scheme above is CPA secure.

\paragraph{Proof:} Suppose for the sake of contradiction that there
exists some time \(T=poly(n)\) adversary \(A\) that succeeds in the
IBE-CPA with probability at least \(1/2+\epsilon\) for some
non-negligible \(\epsilon\). We assume without loss of generality that
whenever \(A\) makes a query to the key distribution function with id
\(id\) or a query to \(H'\) with prefix \(id\), it had already
previously made the query \(id\) to \(H\). (\(A\) can be easily modified
to have this behavior)

We will build an algorithm \(B\) that on input
\(\mathbb{G}_1,\mathbb{G}_2,g,g^a,g^b,g^c\) will output
\(\hat{g}^{abc}\) with probability \(poly(\epsilon,1/T)\).

The algorithm \(B\) will guess \(i_0, j_0 \leftarrow_R \{1,\ldots, T\}\)
and simulate \(A\) ``in its belly'' giving it the public key \(g^a\),
and act as follows:

\begin{itemize}
\item
  When \(A\) makes a query to \(H\) with \(id\), then for all but the
  \(i_0^{th}\) queries, \(B\) will chooose a random
  \(b_{id} \in \{0,\ldots, |\mathbb{G}|\}\) (as usual we'll assume
  \(|\mathbb{G}|\) is prime), choose \(e_{id}=g^{b_{id}}\) and define
  \(H(id)=e_{id}\). Let \(id_0\) be the \(i_0^{th}\) query \(A\) made to
  the oracle. We define \(H(i_0)=g^b\) (where \(g^b\) is the input to
  \(B\)- recall that \(B\) does not know \(b\).)
\item
  When \(A\) makes a query to the key distribution oracle with \(id\)
  then if \(id\neq id_0\) then \(B\) will then respond with
  \(d_{id}=(g^a)^{b_{id}}\). If \(id = id_0\) then \(B\) aborts and
  fails.
\item
  When \(A\) makes a query to the \(H'\) oracle with input
  \(id'\|\hat{h}\) then for all but the \(j_0^{th}\) query \(B\) answers
  with a random string in \(\{0,1\}^\ell\). In the \(j_0^{th}\) query,
  if \(id' \neq id_0\) then \(B\) stops and fails. Otherwise, it outputs
  \(\hat{h}\).
\item
  \(B\) does stops the simulation and fails if we get to the challenge
  part.
\end{itemize}

It might seem weird that we stop the simulation before we reach the
challenge part, but the correctness of this reduction follows from the
following claim:

\paragraph{Claim:} In the actual attack game, with probability at least
\(\epsilon/10\) \(A\) will make the query \(id_*\| \hat{g}^{abc}\) to
the \(H'\) oracle, where \(H(id_*)=g^b\) and the public key is \(g^a\).

\paragraph{Proof:} If \(A\) does not make this query then the message in
the challenge is XOR'ed by a completely random string and \(A\) cannot
distinguish between \(m_0\) and \(m_1\) in this case with probability
better than \(1/2\). QED

Given this claim, to prove the theorem we just need to observe that,
assuming it does not fail, \(B\) provides answers to \(A\) that are
identically distributed to the answers \(A\) receives in an actual
execution of the CPA game, and hence with probability at least
\(\epsilon/(10T^2)\), \(B\) will guess the query \(i_0\) when \(A\)
queries \(H(id_*)\) and set the answer to be \(g^b\), and then guess the
query \(j_0\) when \(A\) queries \(id_*\|\hat{g}^{abc}\) in which case
\(B\)'s output will be correct. QED

\section{Beyond pairing based
cryptography}\label{Beyond-pairing-based-cryptogra}

Boneh and Silverberg asked the question of whether we could go beyond
quadratic polynomials and get schemes that allow us to compute higher
degree. The idea is to get a \emph{multilinear map} which would be a set
of isomorphic groups \(\mathbb{G}_1,\ldots,\mathbb{G}_d\) with
generators \(g_1,\ldots,g_d\) such that we can map \(g_i^a\) and
\(g_j^b\) to \(g_{i+j}^{ab}\).\\
This way we would be able to compute any degree \(d\) polynomial in the
exponent given \(g_1^{x_1},\ldots,g_1^{x_k}\).

We will now show how using such a multilinear map we can get a
construction for a witness encryption scheme. We will only show the
construction, without talking about the security definition, the
assumption, or security reductions.

Given some circuit \(C:\{0,1\}^n\rightarrow\{0,1\}\) and some message
\(x\) we want to ``encrypt'' \(x\) in a way that given \(w\) such that
\(C(w)=1\) it would be possible to decrypt \(x\), and otherwise it
should be hard. It should be noted that the encrypting party itself does
not know any such \(w\) and indeed (as in the case of the proof of
Reimann hypothesis) might not even know if such a \(w\) exists. The idea
is the following. We use the fact that the \texttt{Exact Cover} problem
is NP complete to map \(C\) into collection of subsets
\(S_1,\ldots,S_{m}\) of the universe \(U\) (where \(m,|U|=poly(|C|,n)\))
such that there exists \(w\) with \(C(w)=1\) if and only if there exists
\(d\) sets \(S_{i_1},\ldots,S_{i_d}\) that are a partition of \(U\)
(i.e., every element in \(U\) is covered by exactly one of these sets),
and moreover there is an efficient way to map \(w\) to such a partition
and vice versa. Now, to encrypt the message \(x\) we take a degree \(d\)
instance of multilinear maps
\((\mathbb{G}_1,\ldots,\mathbb{G}_{d},g_1,\ldots,g_{d})\) (with all
groups of size \(p\)) and choose random
\(a_1,\ldots,a_{|U|} \leftarrow_R \{0,\ldots,p-1\}\). We then output the
ciphertext
\(g_1^{\prod_{j\in S_1} a_j},\ldots,g_1^{\prod_{j\in S_m} a_j}, H(g_d^{\prod_{j\in U}^m a_j})\oplus x\).
Now, given a partition \(S_{i_1},\ldots,S_{i_d}\) of the universe \(d\),
we can use the multilinear operations to compute
\(g_d^{\prod_{j\in U}a_j}\) and recover the message. Intuitively, since
the numbers are random, that would be the only way to come up with
computing this value, but showing that requires formulating precise
security definitions for both multilinear maps and witness encryption
and of course a proof.

The first candidate construction for a multilinear map was given by
\href{https://eprint.iacr.org/2012/610}{Garg, Gentry and Halevi}. It is
based on computational questions on lattices and so (perhaps not
surprisingly) it involves significant complications due to \emph{noise}.
At a very high level, the idea is to use a fully homomorphic encryption
scheme that can evaluate polynomials up to some degree \(d\), but
release a ``hobbled decryption key'' that contains just enough
information to provide what's known as a \emph{zero test}: check if an
encryption is equal to zero. Because of the homomorphic properties, that
means that we can check given encryptions of \(x_1,\ldots,x_n\) and some
degree \(d\) polynomial \(P\), whether \(P(x_1,\ldots,x_d)=0\).
Moreover, the notion of security this and similar construction satisfy
is rather subtle and indeed not fully understood. Constructions of
indistinguishability obfuscators are built based on this idea, but are
significantly more involved than the construction of a witness
encryption. One central tool they use is the observation that FHE
reduces the task of obfuscation to essentially obfuscating a decryption
circuit, which can often be rather shallow. But beyond that there is
significant work to be done to actually carry out the obfuscation.
