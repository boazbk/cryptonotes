\chapter{Private key crypto recap}\label{Private-key-crypto-recap}

We now review all that we have learned about \emph{private key}
cryptography before we embark on the wonderful journey to \emph{public
key} cryptography.

This material is mostly covered in Chapters 1 to 9 of the Katz Lindell
book, and now would be a good time for you to read the corresponding
proofs in the book. It is often helpful to see the same proof presented
in a slightly different way. Below is a review of some of the various
reductions we saw in class that are covered in the KL book, with
pointers to the corresponding sections.

\begin{itemize}
\tightlist
\item
  Pseudorandom generators (PRG) length extension (from \(n+1\) output
  PRG to \(poly(n)\) output PRG): Section 7.4.2
\item
  PRG's to pseudorandom functions (PRF's): Section 7.5
\item
  PRF's to Chosen Plaintext Attack (CPA) secure encryption: Section
  3.5.2
\item
  PRF's to secure Message Authentication Codes (MAC's): Section 4.3
\item
  MAC's + CPA secure encryption to chosen ciphertext attack (CCA) secure
  encryption: Section 4.5.4
\item
  Pseudorandom permutation (PRP's) to CPA secure encryption / block
  cipher modes: Section 3.5.2, Section 3.6.2
\item
  Hash function applications: fingerprinting, Merkle trees, passwords:
  Section 5.6
\item
  Coin tossing over the phone: we saw a construction in class that used
  a \emph{commitment scheme} built out of a pseudorandom generator.
  Section 5.6.5 shows an alternative construction using random oracles.
\item
  PRP's from PRF's: we only sketched the construction which can be found
  inSection 7.6
\end{itemize}

One major point we did \emph{not} talk about in this course was
\emph{one way functions}. The definition of a one way function is quite
simple:

A function \(f:\{0,1\}^*\rightarrow\{0,1\}^*\) is a \emph{one way
function} if it is efficiently computable and for every \(n\) and a
\(poly(n)\) time adversary \(A\), the probability over
\(x\leftarrow_R\{0,1\}^n\) that \(A(f(x))\) outputs \(x'\) such that
\(f(x')=f(x)\) is negligible.

The ``OWF conjecture'' is the conjecture that one way functions exist.
It turns out to be a necessary and sufficient condition for much of
cryptography. That is, the following theorem is known (by combining
works of many people):

\paragraph{Theorem:} The following are equivalent: * One way functions
exist * Pseudorandom generators (with non trivial stretch) exist *
Pseudorandom functions exist * CPA secure private key encryptions exist
* CCA secure private key encryptions exist * Message Authentication
Codes exist * Commitment schemes exist

(and others as well)

The key result in the proof of this theorem is the result of Hastad,
Impagliazzo, Levin and Luby that if one way functions exist then
pseudorandom generators exist. If you are interested in finding out
more, Sections 7.2-7.4 in the KL book cover a special case of this
theorem for the case that the one way function is a \emph{permutation}
on \(\{0,1\}^n\) for every \(n\). This proof has been considerably
simplified and quantitatively improved in works of Haitner, Holenstein,
Reingold, Vadhan, Wee and Zheng. See
\href{http://people.seas.harvard.edu/~salil/research/CompEnt-abs.html}{this
talk of Salil Vadhan} for more on this. See also
\href{http://www.cs.princeton.edu/courses/archive/spring08/cos598D/scribe3.pdf}{these
lecture notes} from a Princeton seminar I gave on this topic (though the
proof has been simplified since then by the above works).

\subsection{Attacks on private key
cryptosystems}\label{Attacks-on-private-key-cr}

Another topic we did not discuss in depth is attacks on private key
cryptosystems. These attacks often work by ``opening the black box'' and
looking at the internal operation of block ciphers or hash functions.
One then often assigns variables to various internal registers, and then
we look to finding collections of inputs that would satisfy some
non-trivial relation between those variables. This is a rather vague
description, but you can read KL Section 6.2.6 on \emph{linear} and
\emph{differential} cryptanalysis for more information. See also
\href{http://www.cs.tau.ac.il/~tromer/SKC2006/}{this course of Adi
Shamir}. There is also the fascinating area of \emph{side channel}
attacks on both public and private key crypto.
