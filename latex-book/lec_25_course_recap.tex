\chapter{Course recap}\label{24-Course-recap}

It might be worthwhile to recall what we learned in this course:

\begin{itemize}
\item
  Perhaps first and foremost, that it is possible to
  \emph{mathematically define} what it means for a cryptographic scheme
  to be secure. In the cases we studied, such a definition could always
  be described as a ``security game''. That is, we first define what it
  means for a scheme to be \emph{insecure}. Then, a scheme is secure if
  it is not insecure. The notion of ``insecurity'' is that there exists
  some adversarial strategy that succeeds with higher probability than
  what it should have. We normally don't limit the \emph{strategy} of
  the adversary but only his or her \emph{capabilities}: its
  computational power and the type of access it has to the system (e.g.,
  chosen plaintext, chosen ciphertext, etc.). We also talked how the
  notion of \emph{secrecy} requires \emph{randomness} and how many
  real-life failures of cryptosystems amount to faulty assumptions on
  the sources of randomness.
\item
  We saw the importance of being \emph{conservative} in security
  definitions. For example, how despite the fact that the notion of
  chosen ciphertext attack (CCA) security seems too strong to capture
  any realistic scenario (e.g., when do we let an adversary play with a
  decryption box?), there are many natural cases where the using a CPA
  instead of a CCA secure encryption would lead to an attack on the
  overall protocol.
\item
  We saw how we can prove security by \emph{reductions}. Suppose we have
  a scheme \(S\) that achieves some security notion \(X\) (for example,
  \(S\) might be a function that achieves the security notion of being a
  pseudorandom generator) and we use it to build a scheme \(T\) that we
  want to achieve a security notion \(Y\) (for example, we want \(T\) to
  be a message authentication code). To prove \(T\) is secure, we show
  how we can transform an adversary \(B\) that wins against \(T\) in the
  security game of \(Y\) into an adversary \(A\) that wins against \(S\)
  in the security game of \(X\). Typically, the adversary \(A\) will run
  \(B\) ``in its belly'', simulating for \(B\) the security game \(Y\)
  with respect to \(T\). This can be somewhat confusing, so please
  re-read the last three sentences and make sure you understand this
  crucial notion.
\item
  We also saw some of the concrete wonderful things we can do in
  cryptography:
\item
  In the world of \emph{private key cryptography}, we saw that based on
  the PRG conjecture we can get a CPA secure private key encryption
  (which in particular has key shorter than message), pseudorandom
  functions, message authentication codes, CCA secure encryption,
  commitment schemes, and even zero knowledge proofs for NP complete
  languages.
\item
  We saw that assuming the existence of \emph{collision resistant hash
  functions}, we can get message authentication codes (and digital
  signatures) where the key is shorter than the message. We talked about
  the heuristic of how we can model hash functions as a \emph{random
  oracle} , and use that for ``proofs of work'' in the context of
  bitcoin and password derivation, as well as many other settings.
\item
  We also discussed practical constructions of private key primitives
  such as the AES block ciphers, and how such block ciphers are modeled
  as pseudorandom permutations and how we can use them to get CPA or CCA
  secure encryption via various modes such as CBC or GCM. We also
  discussed the Merkle and Davis-Meyer length extension construction for
  hash functions, and how the Merkle tree construction can be used for
  secure storage.
\item
  We saw the revolutionary notion of \emph{public key encryption}, that
  two people can talk without having coordinated in advance. We saw
  constructions for this based on discrete log (e.g., the Diffie-Hellman
  protocol), factoring (e.g., the Rabin and RSA trapdoor permutations),
  and the \emph{learning with errors} (LWE) problem. We saw the notion
  of digital signatures, and gave several different constructions. We
  saw how we can use digital signatures to create a ``chain of trust''
  via certificates, and how the TLS protocol, which protects web
  traffic, works.
\item
  We talked about some advances notions and in particular saw the
  construction of the surprising concept of a \emph{fully homomorphic
  encryption (FHE)} scheme which has been rightly
  \href{http://bit-player.org/2012/computing-with-encrypted-data}{called
  by Bryan Hayes} ``one of the most amazing magic tricks in all of
  computer science''. Using FHE and zero knowledge proofs, we can get
  multiparty secure computation, which basically means that in the
  setting of interactive protocols between several parties, we can
  establish a ``virtual trusted third party'' (or, as I prefer to call
  it, a ``virtual Chuck Norris'').
\item
  We also saw other variants of encryption such as \emph{functional
  encryption}, \emph{witness encryption} and \emph{identity based
  encryption}, which allow for ``selective leaking'' of information. For
  functional encryption and witness encryption we don't yet have clean
  constructions under standard assumptions but only under obfuscation,
  but we saw how we could get identity based encryption using the random
  oracle heuristic and the assumption of the difficulty of the discrete
  logarithm problem in a group that admits an efficient \emph{pairing}
  operation.
\item
  We talked about the notion of obfuscation, which can be thought as the
  one tool that, if exists, would imply all the others. We saw that
  virtual black box obfuscation does not exist, but there might exist a
  weaker notion known as ``indistinguishability obfuscation'' and we saw
  how it can be useful via the example of a witness encryption and a
  digital signature scheme. We mentioned (without proof) that it can
  also be used to obtain a functional encryption scheme.
\item
  We talked about how quantum computing can change the landscape of
  cryptography, making lattice based constructions our main candidate
  for public key schemes.
\item
  Finally, we discussed some of the ethical and policy issues that arise
  in the applications of cryptography, and what is the impact
  cryptography has now, or can have in the future, on society.
\end{itemize}

\section{Some things we did not cover}\label{24-Some-things-we-did-not}

\begin{itemize}
\item
  We did not cover what is arguably the other ``fundamental theorem of
  cryptography'', namely the equivalence of one-way functions and
  pseudorandom generators. A one-way function is an efficient map
  \(F:\{0,1\}^*\rightarrow\{0,1\}^*\) that is hard to invert on a random
  input. That is, for any efficient algorithm \(A\) if \(A\) is given
  \(y=F(x)\) for uniformly chosen \(x\leftarrow_R\{0,1\}^n\), then the
  probability that \(A\) outputs \(x'\) with \(F(x')=y\) is negligible.
  It can be shown that one-way functions are \emph{minimal} in the sense
  that they are \emph{necessary} for a great many cryptographic
  applications including pseudorandom generators and functions,
  encryption with key shorter than the message, hash functions, message
  authentication codes, and many more. (Most of these results are
  obtained via the work of Impagliazzo and Luby who showed that if
  one-way functions do not exist then there is a \emph{universal
  posterior sampler} in the sense that for every probabilistic process
  \(F\) that maps \(x\) to \(y\), there is an efficient algorithm that
  given \(y\) can sample \(x'\) from a distribution close to the
  posterior distribution of \(x\) conditioned on \(F(x)=y\). This result
  is typically known as the equivalence of standard one-way functions
  and distributional one-way functions.) The fundamental result of
  Hastad, Impagliazzo, Levin and Luby is that one-way functions are also
  \emph{sufficient} for much of private key cryptography since they
  imply the existence of pseudorandom generators.
\item
  Related to this, although we mentioned this briefly, we did not go in
  depth into ``Impagliazzo's Worlds'' of algorithmica, heuristica,
  pessiland, minicrypt, cryptomania (and the new one of ``obfustopia'').
  If this piques your curiosity, please read
  \href{http://www.cs.mun.ca/~kol/courses/6743-w15/papers/russell-fiveworlds.pdf}{this
  1995 survey}.
\item
  We did not go in detail into the design of private key cryptosystems
  such as the AES. Though we discussed modes of operation of block
  ciphers, we did not go into a full description of all modes that are
  used in practice. We also did not discuss cryptanalytic techniques
  such as linear and differential cryptanalysis. We also not discuss all
  technical issues that arise with length extension and padding of
  encryptions in practice.
\item
  While we talked about bitcoin, the TLS protocol, two factor
  authentication systems, and some aspects of pretty good privacy, we
  restricted ourselves to abstractions of these systems and did not
  attempt a full ``end to end'' analysis of a complete system. I do hope
  you have learned the tools that you'd be able to understand the full
  operation of such a system if you need to.
\item
  While we talked about Shor's algorithm, the algorithm people actually
  use today to factor numbers is the \emph{number field sieve}. It and
  its predecessor, the quadratic sieve, are well worth studying. The
  (freely available online) \href{http://www.shoup.net/ntb/}{book of
  Shoup} is an excellent source not just for these algorithms but
  general algorithmic group/number theory.
\item
  We talked about some attacks on practical systems, but there many
  other attacks that teach us important lessons, not just about these
  particular systems, but also about security and cryptography in
  general (as well some human tendencies to repeat certain types of
  mistakes).
\end{itemize}

\section{What I hope you learned}\label{24-What-I-hope-you-learne}

I hope you got an appreciation for cryptography, and an understanding of
how it can surprise you both in the amazing security properties it can
deliver, as well in the subtle, but often devastating ways, that it can
fail. Beyond cryptography, I hope you got out of this course the ability
to think a little differently - to be paranoid enough to see the world
from the point of view of an adversary, but also the lesson that
sometimes if something sounds crazy but is not downright impossible it
might just be feasible.\\
But if these philosophical ramblings don't speak to you, as long as you
know the difference between CPA and CCA and I won't catch you reusing a
one-time pad, you should be in good shape :)

I did not intend this course to teach you how to implement cryptographic
algorithms, but I do hope that if you need to use cryptography at any
point, you now have the skills to read up what's needed and be able to
argue intelligently about the security of real-world systems. I also
hope that you have now sufficient background to not be scared by the
technical jargon and the abundance of adjectives in cryptography
research papers and be able to read up on what you need to follow any
paper that is interesting to you.

Mostly, I just hope you enjoyed this last term and felt like this course
was a good use of your time. I certainly did.
