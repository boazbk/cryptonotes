\chapter{Ethical, moral, and policy dimensions to
cryptography}\label{23-Ethical-moral-and-poli}

This will not be a lecture but rather a discussion on some of the
questions that arise from cryptography. I would like you to read some of
the sources below (and maybe others) and reflect on the following
questions:

The discussion is often framed as weighing privacy against security, but
I encourage you to look critically at both issues. It is often
instructive to try to compare the current situation with both the
historical past as well as some ideal desired world. It is also
worthwhile to consider cryptography in the broader contexts. Some people
on both the pro regulation and anti regulation camps exeggarate the role
of cryptography.\\
On one hand, cryptography is likely not to bring about the ``crypto
anarchy'' regime hoped for in the crypto anarchist manifesto. For
example, more than the growth of bitcoin, we are seeing a turn away from
cash into credit cards and other forms of much more traceable and
\emph{less} anonymous forms of payments (interestingly, these forms of
payments are often enabled by cryptography). On the other hand, despite
the fears raised by government agencies of ``going dark'' there are
powerful commercial incentives to collect vast amounts of data and store
them at search-warrant friendly servers. Clearly technology is shifting
the landscape of relationships among individuals, as well as between
individuals and large organizations and governments. Cryptography is an
important component in these technologies but not the only one, and more
than that, the ways technologies end up \emph{used} often has more to do
with social and commercial factors than with the technologies
themselves.

All that said, significant changes often pose non trivial dangers, and
it is important to have an informed and reasoned discussion of the ways
cryptography can help or harm the general and private good.

Some questions that are worth considering are:

\begin{itemize}
\item
  Is communicating privately a basic
  \href{http://www.un.org/en/universal-declaration-human-rights/}{human
  right}? Should it extend to communicating at a distance? Should this
  be absolute privacy that cannot be violated even with a legal warrant?
  If there was a secure way to implement wiretapping only with a legal
  warrant, would it be morally just?
\item
  Is privacy a basic good in its own right? Or a necessary condition for
  the freedom of expression, and peaceful assembly and association?
\item
  Are we less or more secure today than in the past? In what ways did
  the balance between government and individuals shift in the last few
  decades? Do governments have more or less data and tools for
  monitoring individuals at their disposal? Do individuals and
  non-governmental groups have more or less ability to inflict harm (and
  hence need to be protected against)?
\item
  Do we have more or less privacy today than in the past? Do
  cryptography regulation play a big part in that?
\item
  What would be the balance between security and privacy in an ideal
  world?
\item
  Is the focus on encryption misguided in that the main issue affecting
  privacy and security is the so called \emph{meta data}? Can
  cryptographic techniques protect such meta data? Even if they could,
  is there a commercial interest in doing so?
\item
  One argument against the regulation of cryptography is that, given the
  mathematics of cryptography is not secret, the ``bad guys'' will
  always be able to access it. Is this a valid argument? Note that
  similar arguments are made in the context of gun control. Also,
  perhaps the ``true dissidents'' will also be able to access
  cryptography as well and so regulation will effect the masses or ``run
  of the mill'' private good and not-so-good citizens?
\item
  What would be the practical impact of regulations forbidding the use
  of end-to-end crypto without access by governments?
\item
  Rogaway argues that cryptography is inherently political, and research
  should acknowledge this and be directed at achieving beneficial
  political goals. Has cryptography research failed the public? What
  more could be done?
\item
  Are some cryptographic (or crypto related) tools inherently morally
  problematic? Rogaway suggests that this may be true for fully
  homomorphic encryption and differential privacy- do you agree?
\item
  What are the most significant scenarios where cryptography can impact
  positively or negatively? Large scale terror attacks? ``Ordinary''
  crimes (that still claim the lives of many more people than terror
  attacks)? Attacks against cyber infrastructure or personal data?
  Political dissidents in opressive regimes? Mass government or
  corporate surveilance?
\item
  How are these issues different in the U.S. as opposed to other
  countries? Is the debate too U.S. centric?
\end{itemize}

\section{Reading prior to lecture:}\label{23-Reading-prior-to-lectu}

\begin{itemize}
\tightlist
\item
  \href{http://web.cs.ucdavis.edu/~rogaway/papers/moral.html}{Moral
  Character of Cryptographic Work} - please read at least parts 1-3
  (pages 1-30 in the footnoted version) - it's long and should not be
  taken uncritically, but is a very good and thought provoking read.
\item
  \href{http://cyber.law.harvard.edu/pubrelease/dont-panic/}{``Going
  Dark'' Berkman report} - this is a report written by a committee, and
  as such not as exciting (though arguably more sober) than Rogaway's
  paper. Please read at least the introduction and you might also find
  the personal statements in Appendix A interesting.
\item
  \href{https://www.digitalequilibriumproject.com/home.html}{Digital
  Equilibrium project} - optional reading - this is a group of very
  senior current and former officials, in particular in government, and
  as such would tend to fall on the more ``establishment'' or ``pro
  regulation'' side. Their ``foundational paper'' has even more of a
  ``written by committee'' feel but is still worthwhile reading.
\item
  \href{http://www.activism.net/cypherpunk/crypto-anarchy.html}{Crypto
  anarchist manifesto} - optional reading - very much not ``written by
  committee'' can be an interesting read even if it sounds more like
  science fiction than describing actual current or near future reality.
\end{itemize}

\section{Case studies.}\label{23-Case-studies}

Since such a discussion might be sometimes hard to hold in the abstract,
let us consider some actual cases:

\subsection{The Snowden revelations}\label{23-The-Snowden-revelation}

The impetus for the current iteration of the security vs privacy debate
were the \href{http://www.theguardian.com/us-news/the-nsa-files}{Snowden
revelations} on the massive scale of surveillance by the NSA on citizens
in the U.S. and around the globe. Concurrently, in plain sight,
companies such as Apple, Google, Facebook, and others are also
collecting massive amounts of information on their users. Some of the
backlash to the Snowden revelations was increased pressure on companies
to support stronger ``end-to-end'' encryption such as some data does not
reside on companies' servers, that have become suspect. We're now seeing
some ``backlash to the backlash'' with law enforcement and government
officials around the globe trying to ban such encryption technlogoy or
mandate government backdoors.

\subsection{FBI vs Apple case}\label{23-FBI-vs-Apple-case}

We've mentioned
\href{http://www.npr.org/sections/thetwo-way/2016/02/17/467096705/apple-the-fbi-and-iphone-encryption-a-look-at-whats-at-stake}{this
case} in the past. (I also
\href{https://windowsontheory.org/2016/03/04/the-iphones-of-terrorists/}{blogged}
about it.) The short summary is that an iPhone belonging to one of the
San Bernardino terrorists was found by the FBI. The iPhone's memory was
encrypted by a key \(k\) that is obtained as \(H(uid\|passcode)\) where
\(passcode\) is the six digit passcode of the user and \(uid\) is a
secret \(128\) bit key that is hardwired into the processor. The
processor will only allow ten attempts at guessing the passcode before
erasing all memory. The FBI wanted Apple's help in creating a digitally
signed software update that essentially run a brute force search over
the \(10^6\) passcodes and output the key \(k\). The software update
could be restricted to run only on that particular iPhone. Eventually,
the FBI managed to extract the information out of the iPhone without
Apple's help. The method they used is unknown, but it may be possible to
physically extract the \(uid\) from the processor. It might also be
possible to prevent erasure of the memory by disconnecting it from the
processor, or rewriting it after erasure. Would such cases change your
position on this question?

Some questions that one could ask:

\begin{itemize}
\item
  Given that the FBI had a legal warrant for the information on the
  iPhone, was it wrong of Apple to refuse to provide the help required?
\item
  Was it wrong for Apple to have designed their iPhone so that they are
  unable to easily extract information out of it? Should they be
  required to make sure that such devices can be searched as a result of
  a legal warrant?
\item
  If the only way for the FBI to get the information was to get Apple's
  master signature key (that allows to completely break into any iPhone,
  and even turn it into a recording/surveillance device), would it have
  been OK for them to do it? Should Apple design their device in a way
  that even their master signature key cannot break them? Is that even
  possible, given that software updates are crucial for proper
  functioning of such devices? (It was recently
  \href{http://www.theverge.com/2016/4/14/11434926/blackberry-encryption-master-key-broken-canada-rcmp-surveillance}{claimed}
  that Canadian police has had access to the master decryption key of
  Blackberry since 2010.)
\end{itemize}

In the San Bernardino case, the utility of breaking into the phone was
questioned, given that both perpetrators were killed and there was no
evidence of them receiving any assistance. But there are cases where
things are more complicated.
\href{http://www.npr.org/sections/alltechconsidered/2016/03/30/472302719/mom-asks-who-will-unlock-her-murdered-daughters-iphone}{Brittney
Mills} was 29 years old and 8 months pregnant when she was shot and
killed in April 2015 in Baton Rouge, Louisiana. Her baby was delivered
via emergency C section but also died a week later. There was no sign of
forced entry and so it is quite likely she knew her assailant. Her
family believes that the clues to her murderer's identity could be found
in her iPhone, but since it is locked they have no way of extracting
this information. One can imagine other cases as well. Recently a mother
found her kidnapped daughter using the
\href{http://www.nbcnews.com/news/us-news/find-my-iphone-icloud-lead-cops-kidnapped-teen-n521486}{Find
my iPhone} procedure. It is not hard to concieve of a case where
unlocking a phone is the key to saving someone's life. Would such cases
change your view of the above questions?

\subsection{Juniper backdoor case and the OPM
break-in}\label{23-Juniper-backdoor-case-}

We've also mentioned the case of the
\href{http://www.wired.com/2015/12/juniper-networks-hidden-backdoors-show-the-risk-of-government-backdoors/}{Juniper
backdoor case}. This was a break in to the firewalls of Juniper networks
by an unknown party that was crucially enabled by backdoor allegedly
inserted by the NSA into the Dual EC pseudorandom generator. (see also
\href{https://rpw.sh/blog/2015/12/21/the-backdoored-backdoor/}{here} and
\href{http://blog.cryptographyengineering.com/2015/12/on-juniper-backdoor.html}{here}
for more).

Because of the nature of this break in, whomever is responsible for it
could have decrypted much of the traffic without leaving any traces, and
so we don't know the damage caused, but such hacks can have much more
significant consequences than forcing people to change their credit card
numbers. When the
\href{https://www.lawfareblog.com/why-opm-hack-far-worse-you-imagine}{federal
office of personell management was hacked} sensitive information about
millions of people who have gone through the security clearance was
extracted. This includes fingerprints, extensive personal information
from interviews and polygraph sessions, and much more. Such information
can help then gain access to more information, whether it's using the
fingerprint to unlock a phone or using the extensive knowledge of social
connections, habits and interests to launch very targeted attacks to
extract information from particular individuals.

Here one could ask if stronger cryptography, and in particular
cryptographic tools that would have enabled an individual to control
access to his or her own data, would have helped prevent such attacks.
