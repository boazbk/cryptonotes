\chapter{Establishing secure connections over insecure
channels}\label{14-Establishing-secure-co}

We've now compiled all the tools that are needed for the basic goal of
cryptography (which is still being subverted quite often) allowing Alice
and Bob to exchange messages assuring their integrity and
confidentiality over a channel that is observed or controlled by an
adversary. Our tools for achieving this goal are:

\begin{itemize}
\item
  Public key (aka asymmetric) encryption schemes.
\item
  Public key (aka asymmetric) digital signatures schemes.
\item
  Private key (aka symmetric) encryption schemes - block ciphers and
  stream ciphers.
\item
  Private key (aka symmetric) message authentication codes and
  pseudorandom functions.
\item
  Hash functions that are used both as ways to compress messages for
  authentication as well as key derivation and other tasks.
\end{itemize}

The notions of security we require from these building blocks can vary
as well. For encryption schemes we talk about CPA (chosen plaintext
attack) and CCA (chosen ciphertext attacks), for hash functions we talk
about collision-resistance, being used (combined with keys) as
pseudorandom functions, and then sometimes we simply model those as
random oracles. Also, all of those tools require access to a source of
randomness, and here we use hash functions as well for entropy
extraction.

\section{Cryptography's obsession with
adjectives.}\label{14-Cryptographys-obsessio}

As we learn more and more cryptography we see more and more adjectives,
every notion seems to have modifiers such as ``non-malleable'',
``leakage-resilient'', ``identity based'', ``concurrently secure'',
``adaptive'', ``non-interactive'', etc.. etc\ldots{} . Indeed, this
motivated a parody web page of an
\href{https://cseweb.ucsd.edu/~mihir/crypto-topic-generator.html}{automatic
crypto paper title generator}. Unlike algorithms, where typically there
are straightforward \emph{quantitative} tradeoffs (e.g., faster is
better), in cryptography there are many \emph{qualitative} ways
protocols can vary based on the assumptions they operate under and the
notions of security they provide.

In particular, the following issues arise when considering the task of
securely transmitting information between two parties Alice and Bob:

\begin{itemize}
\item
  \textbf{Infrastructure/setup assumptions:} What kind of setup can
  Alice and Bob rely upon? For example in the TLS protocol, typically
  Alice is a website and Bob is user; Using the infrastructure of
  certificate authorities, Bob has a trusted way to obtain Alice's
  \emph{public signature key}, while Alice doesn't know anything about
  Bob. But there are many other variants as well. Alice and Bob could
  share a (low entropy) \emph{password}. One of them might have some
  hardware token, or they might have a secure out of band channel (e.g.,
  text messages) to transmit a short amount of information. There are
  even variants where the parties authenticate by something they
  \emph{know}, with one recent example being the notion of \emph{witness
  encryption} (Garg, Gentry, Sahai, and Waters) where one can encrypt
  information in a ``digital time capsule'' to be opened by anyone who,
  for example, finds a proof of the Riemann hypothesis.
\item
  \textbf{Adversary access:} What kind of attacks do we need to protect
  against. The simplest setting is a \emph{passive} eavesdropping
  adversary (often called ``Eve'') but we sometimes consider an
  \emph{active person-in-the-middle} attacks (sometimes called
  ``Mallory''). We sometimes consider notions of \emph{graceful
  recovery}. For example, if the adversary manages to hack into one of
  the parties then it can clearly read their communications from that
  time onwards, but we would want their past communication to be
  protected (a notion known as \emph{forward secrecy}). If we rely on
  trusted infrastructure such as certificate authorities, we could ask
  what happens if the adversary breaks into those. Sometimes we rely on
  the security of several entities or secrets, and we want to consider
  adversaries that control \emph{some} but not \emph{all} of them, a
  notion known as \emph{threshold cryptography}.
\item
  \textbf{Interaction:} Do Alice and Bob get to interact and relay
  several messages back and forth or is it a ``one shot'' protocol? You
  may think that this is merely a question about efficiency but it turns
  out to be crucial for some applications. Sometimes Alice and Bob might
  not be two parties separated in space but the same party separated in
  time. That is, Alice wishes to send a message to her future self by
  storing an encrypted and authenticated version of it on some media. In
  this case, absent a time machine, back and forth interaction between
  the two parties is obviously impossible.
\item
  \textbf{Security goal:} The security goals of a protocol are usually
  stated in the negative- what does it mean for an adversary to
  \emph{win} the security game. We typically want the adversary to learn
  absolutely no information about the secret beyond what she obviously
  can. For example, if we use a shared password chosen out of \(t\)
  possibilities, then we might need to allow the adversary \(1/t\)
  success probability, but we wouldn't want her to get anything beyond
  \(1/t+negl(n)\). In some settings, the adversary can obviously
  completely disconnect the communication channel between Alice and Bob,
  but we want her to be essentially limited to either dropping
  communication completely or letting it go by unmolested, and not have
  the ability to modify communication without detection. Then in some
  settings, such as in the case of steganography and anonymous routing,
  we would want the adversary not to find out even the fact that a
  conversation had taken place.
\end{itemize}

\section{Basic Key Exchange protocol}\label{14-Basic-Key-Exchange-pro}

The basic primitive for secure communication is a \emph{key exchange}
protocol, whose goal is to have Alice and Bob share a common random
secret key \(k\in\{0,1\}^n\). Once this is done, they can use a CCA
secure / authenticated private-key encryption to communicate with
confidentiality and integrity.

The canonical example of a basic key exchange protocol is the
\emph{Diffie Hellman} protocol. It uses as public parameters a group
\(\mathbb{G}\) with generator \(g\), and then follows the following
steps:

\begin{enumerate}
\def\labelenumi{\arabic{enumi}.}
\item
  Alice picks random \(a\leftarrow_R\{0,\ldots,|\mathbb{G}|-1\}\) and
  sends \(A=g^a\).
\item
  Bob picks random \(b\leftarrow_R \{0,\ldots,|\mathbb{G}|-1\}\) and
  sends \(B=g^b\).
\item
  They both set their key as \(k=H(g^{ab})\) (which Alice computes as
  \(B^a\) and Bob computes as \(A^b\)), where \(H\) is some hash
  function.
\end{enumerate}

Another variant is using an arbitrary public key encryption scheme such
as RSA:

\begin{enumerate}
\def\labelenumi{\arabic{enumi}.}
\item
  Alice generates keys \((d,e)\) and sends \(e\) to Bob.
\item
  Bob picks random \(k \leftarrow_R\{0,1\}^m\) and sends \(E_e(k)\) to
  Alice.
\item
  They both set their key to \(k\) (which Alice computes by decrypting
  Bob's ciphertext)
\end{enumerate}

Under plausible assumptions, it can be shown that these protocols secure
against a \emph{passive} eavesdropping adversary Eve. The notion of
security here means that, similar to encryption, if after observing the
transcript Eve receives with probability \(1/2\) the value of \(k\) and
with probability \(1/2\) a random string \(k'\gets\{0,1\}^n\), then her
probability of guessing which is the case would be at most
\(1/2+negl(n)\) (where \(n\) can be thought of as \(\log |\mathbb{G}|\)
or some other parameter related to the length of bit representation of
members in the group).

\section{Authenticated key exchange}\label{14-Authenticated-key-exch}

The main issue with this key exchange protocol is of course that
adversaries often are \emph{not} passive. In particular, an active Eve
could agree on her own key with Alice and Bob separately and then be
able to see and modify all future communication. She might also be able
to create weird (with some potential security implications) correlations
by, say, modifying the message \(A\) to be \(A^2\) etc..

For this reason, in actual applications we typically use
\emph{authenticated} key exchange. The notion of authentication used
depends on what we can assume on the setup assumptions. A standard
assumption is that Alice has some public keys but Bob doesn't. (This is
the case when Alice is a website and Bob is a user.) However, one needs
to take care in how to use this assumption. Indeed, the standard
protocol for securing the web: the
\href{https://goo.gl/md9Bsa}{transport Layer Security (TLS) protocol}
(and its predecessor SSL) has gone through six revisions (including a
name change from SSL to TLS) largely because of security concerns. We
now illustrate one of those attacks.

\subsection{Bleichenbacher's attack on RSA PKCS V1.5 and SSL
V3.0}\label{14-Bleichenbachers-attack}

If you have a public key, a natural approach is to take the
encryption-based protocol and simply skip the first step since Bob
already knows the public key \(e\) of Alice. This is basically what
happened in the SSL V3.0 protocol. However, as was
\href{http://archiv.infsec.ethz.ch/education/fs08/secsem/bleichenbacher98.pdf}{shown
by Bleichenbacher in 1998}, it turns out this is susceptible to the
following attack:

\begin{itemize}
\item
  The adversary listens in on a conversation, and in particular observes
  \(c=E_e(k)\) where \(k\) is the private key.
\item
  The adversary then starts many connections with the server with
  ciphertexts related to \(c\), and observes whether they succeed or
  fail (and in what way they fail, if they do). It turns out that based
  on this information, the adversary would be able to recover the key
  \(k\).
\end{itemize}

Specifically, the version of RSA (known as PKCS \(\sharp\) V1.5) used in
the SSL V3.0 protocol requires the value \(x\) to have a particular
format, with the top two bytes having a certain form. If in the course
of the protocol, a server decrypts \(y\) and gets a value \(x\) not of
this form then it would send an error message and halt the connection.
While the designers of SSL V3.0 might not have thought of it that way,
this amounts to saying that an SSL V3.0 server supplies to any party an
oracle that on input \(y\) outputs \(1\) iff \(y^{d} \pmod{m}\) has this
form, where \(d = e^{-1} \pmod|\Z^*_m|\) is the secret decryption key.
It turned out that one can use such an oracle to invert the RSA
function. For a result of a similar flavor, see the (1/2 page) proof of
Theorem 11.31 (page 418) in KL, where they show that an oracle that
given \(y\) outputs the least significant bit of \(y^d \pmod{m}\) allows
to invert the RSA function.\footnote{The first attack of this flavor was
  given in the 1982 paper of Goldwasser, Micali, and Tong.
  Interestingly, this notion of ``hardcore bits'' has been used for both
  practical \emph{attacks} against cryptosystems as well as theoretical
  (and sometimes practical) \emph{constructions} of other cryptosystems.}

For this reason, new versions of the SSL used a different variant of RSA
known as PKCS \$\sharp\$1 V2.0 which satisfies (under assumptions)
\emph{chosen ciphertext security (CCA)} and in particular such oracles
cannot be used to break the encryption. (Nonetheless, there are still
some implementation issues that allowed to perform some attacks, see the
note in KL page 425 on Manfer's attack.)

\section{Chosen ciphertext attack security for public key
cryptography}\label{14-Chosen-ciphertext-atta}

The concept of chosen ciphertext attack security makes perfect sense for
\emph{public key} encryption as well. It is defined in the same way as
it was in the private key setting:

\hypertarget{CCSpubdef}{}
\begin{definition}[CCA secure public key encryption] \label[definition]{CCSpubdef}

A public key encryption scheme \((G,E,D)\) is \emph{chosen ciphertext
attack (CCA) secure} if every efficient Mallory wins in the following
game with probability at most \(1/2+ negl(n)\):

\begin{itemize}
\item
  The keys \((e,d)\) are generated via \(G(1^n)\), and Mallory gets the
  public encryption key \(e\) and \(1^n\).
\item
  For \(poly(n)\) rounds, Mallory gets access to the function
  \(c \mapsto D_d(c)\). (She doesn't need access to \(m \mapsto E_e(m)\)
  since she already knows \(e\).)
\item
  Mallory chooses a pair of messages \(\{ m_0,m_1 \}\), a secret \(b\)
  is chosen at random in \(\{0,1\}\), and Mallory gets
  \(c^* = E_e(m_b)\). (Note that she of course does \emph{not} get the
  randomness used to generate this challenge encryption.)
\item
  Mallory now gets another \(poly(n)\) rounds of access to the function
  \(c \mapsto D_d(c)\) except that she is not allowed to query \(c^*\).
\item
  Mallory outputs \(b'\) and \emph{wins} if \(b'=b\).
\end{itemize}

\end{definition}

In the private key setting, we achieved CCA security by combining a
CPA-secure private key encryption scheme with a message authenticating
code (MAC), where to CCA-encrypt a message \(m\), we first used the
CPA-secure scheme on \(m\) to obtain a ciphertext \(c\), and then added
an authentication tag \(\tau\) by signing \(c\) with the MAC. The
decryption algorithm first verified the MAC before decrypting the
ciphertext. In the public key setting, one might hope that we could
repeat the same construction using a CPA-secure \emph{public key}
encryption and replacing the MAC with \emph{digital signatures}.

\begin{pause} \label[pause]{14-Try-to-think-what-woul}

Try to think what would be such a construction, and whether there is a
fundamental obstacle to combining digital signatures and public key
encryption in the same way we combined MACs and private key encryption.

\end{pause}

Alas, as you may have realized, there is a fly in this ointment. In a
signature scheme (necessarily) it is the \emph{signing key} that is
\emph{secret}, and the \emph{verification key} that is \emph{public}.
But in a public key encryption, the \emph{encryption} key is
\emph{public}, and hence it makes no sense for it to use a secret
signing key. (It's not hard to see that if you reveal the secret signing
key then there is no point in using a signature scheme in the first
place.)

\paragraph{Why CCA security matters.} For the reasons above,
constructing CCA secure public key encryption is very challenging. But
is it worth the trouble? Do we really need this ``ultra conservative''
notion of security? The answer is \emph{yes}. Just as we argued for
\emph{private key} encryption, chosen ciphertext security is the notion
that gets us as close as possible to designing encryptions that fit the
metaphor of \emph{secure sealed envelopes}. Digital analogies will never
be a perfect imitation of physical ones, but such metaphors are what
people have in mind when designing cryptographic protocols, which is a
hard enough task even when we don't have to worry about the ability of
an adversary to reach inside a sealed envelope and XOR the contents of
the note written there with some arbitrary string. Indeed, several
practical attacks, including Bleichenbacher's attack above, exploited
exactly this gap between the physical metaphor and the digital
realization. For more on this, please see
\href{http://www.shoup.net/papers/expo.pdf}{Victor Shoup's survey} where
he also describes the Cramer-Shoup encryption scheme which was the first
practical public key system to be shown CCA secure without resorting to
the random oracle heuristic. (The first definition of CCA security, as
well as the first polynomial-time construction, was given in a seminal
1991 work of Dolev, Dwork and Naor.)

\section{CCA secure public key encryption in the Random Oracle
Model}\label{14-CCA-secure-public-key-}

We now show how to convert any CPA-secure public key encryption scheme
to a CCA-secure scheme in the random oracle model (this construction is
taken from Fujisaki and Okamoto, CRYPTO 99). In the homework, you will
see a somewhat simpler direct construction of a CCA secure scheme from a
\emph{trapdoor permutation}, a variant of which is known as OAEP (which
has better ciphertext expansion) has been standardized as PKCS
\$\sharp\$1 V2.0 and is used in several protocols. The advantage of a
generic construction is that it can be instantiated not just with the
RSA and Rabin schemes, but also directly with Diffie-Hellman and Lattice
based schemes (though there are direct and more efficient variants for
these as well).

\begin{quote} \label[quote]{14-CCA-ROM-ENC-SchemeIngr}

\textbf{CCA-ROM-ENC Scheme:}

\begin{itemize}
\item
  \textbf{Ingredients:} A public key encryption scheme \((G',E',D')\)
  and a two hash functions \(H,H':\{0,1\}^*\rightarrow\{0,1\}^n\) (which
  we model as independent random oracles\footnote{Recall that it's easy
    to obtain two independent random oracles \(H,H'\) from a single
    oracle \(H''\), for example by letting \(H(x)=H''(0\|x)\) and
    \(H'(x)=H''(1\|x)\).})
\item
  \textbf{Key generation:} We generate keys \((e,d)=G'(1^n)\) for the
  underlying encryption scheme.
\item
  \textbf{Encryption:} To encrypt a message \(m\in\{0,1\}^\ell\), we
  select randomness \(r\leftarrow_R\{0,1\}^\ell\) for the underlying
  encryption algorithm \(E'\) and output
  \begin{equation*}
  E_e(m)= E'_e(r;H(m\|r))\|(H'(r) \oplus m)\;,
  \end{equation*}
  where by \(E'_e(m';r')\) we denote the result of encrypting the
  plaintext \(m'\) using the key \(e\) and the randomness \(r'\) (we
  assume the scheme takes \(n\) bits of randomness as input; otherwise
  modify the output length of \(H\) accordingly).
\item
  \textbf{Decryption:} To decrypt a ciphertext \(c\|y\) first let
  \(r=D'_d(c)\), \(m=H'(r) \oplus y\) and then check that
  \(c=E'_e(m;H(m\|r))\). If the check fails we output \texttt{error};
  otherwise we output \(m\).
\end{itemize}

\end{quote}

\hypertarget{CCAPKCthm}{}
\begin{theorem}[CCA security from random oracles] \label[theorem]{CCAPKCthm}

The above CCA-ROM-ENC scheme is CCA secure.

\end{theorem}

\begin{proof} \label[proof]{14-Note-The-proof-here-re}

\textbf{Note:} The proof here refers to the original scheme in the notes
(which was not secure) - should be updated by the scribes to the correct
proof as presented in lecture.

Suppose towards a contradiction that there exists an adversary \(M\)
that wins the CCA game with probability at least \(1/2+\epsilon\) where
\(\epsilon\) is non-negligible. Our aim is to show that the decryption
box would be ``useless'' to \(M\) and hence reduce CCA security to CPA
security (which we'll then derive from the CPA security of the
underlying scheme).

Consider the following ``box'' \(\hat{D}\) that will answer decryption
queries \(c\|y\|z\) of the adversary as follows:\\
* If \(z\) was returned before to the adversary as an answer to
\(H'(m\|r)\) for some \(m,r\), and \(c=E'_e(m\;H(m\|r))\) and
\(y=m\oplus r\) then return \(m\).\\
* Otherwise return \texttt{error}

\textbf{Claim:} The probability that \(\hat{D}\) answers a query
differently then \(D\) is negligible.

\textbf{Proof of claim:} If \(D\) gives a non \texttt{error} response to
a query \(c\|y\|z\) then it must be that \(z=H'(m\|r)\) for some \(m,r\)
such that \(y = r\oplus m\) and \(c=E_e(r;H(m\|r))\), in which case
\(D\) will return \(m\). The only way that \(\hat{D}\) will answer this
question differently is if \(z=H'(m\|r)\) but the query \(m\|r\) hasn't
been asked before by the adversary. Here there are two options. If this
query has never been asked before at all, then by the lazy evaluation
principle in this case we can think of \(H'(m\|r)\) as being
independently chosen at this point, and the probability it happens to
equal \(z\) will be \(2^{-n}\). If this query was asked by someone apart
from the adversary then it could only have been asked by the encryption
oracle while producing the challenge ciphertext \(c^*\|y^*\|z^*\), but
since the adversary is not allowed to ask this precise ciphertext, then
it must be a ciphertext of the form \(c\|y\|z^*\) where
\((c,y) \neq (c^*,y^*)\) and such a ciphertext would get an
\texttt{error} response from both oracles. \textbf{QED (claim)}

Note that we can assume without loss of generality that if \(m^*\) is
the challenge message and \(r^*\) is the randomness chosen in this
challenge, the adversary never asks the query \(m^*\|r^*\) to the its
\(H\) or \(H'\) oracles, since we can modify it so that before making a
query \(m\|r\), it will first check if \(E_e(m\;r)=c^*\) where
\(c^*\|y^*\|z^*\) is the challenge ciphertext, and if so use this to win
the game.

In other words, if we modified the experiment so the values
\(R^*=H(r^*\|m)\) and \(z^*=H'(m^*\|r^*)\) chosen while producing the
challenge are simply random strings chosen completely independently of
everything else. Now note that our oracle \(\hat{D}\) did \emph{not}
need to use the decryption key \(d\). So, if the adversary wins the CCA
game, then it wins the \emph{CPA game} for the encryption scheme
\(E_e(m) = E'_e(r;R)\| r \oplus m \| R'\) where \(R\) and \(R'\) are
simply independent random strings; we leave proving that this scheme is
CPA secure as an exercise to the reader.

\end{proof}

\subsection{Defining secure authenticated key
exchange}\label{14-Defining-secure-authen}

The basic goal of secure communication is to set up a \emph{secure
channel} between two parties Alice and Bob. We want to do so over an
open network, where messages between Alice and Bob might be read,
modified, deleted, or added by the adversary. Moreover, we want Alice
and Bob to be sure that they are talking to one another rather than
other parties. This raises the question of what is identity and how is
it verified. Ultimately, if we want to use identities, then we need to
trust some authority that decides which party has which identity. This
is typically done via a \emph{certificate authority (CA)}. This is some
trusted authority, whose verification key \(v_{CA}\) is public and known
to all parties. Alice proves in some way to the CA that she is indeed
Alice, and then generates a pair \((s_{Alice},v_{Alice})\), and gets
from the CA the message \(\sigma_{Alice}\)=``The key \(v_{Alice}\)
belongs to Alice'' signed with \(s_{CA}\).\footnote{The registration
  process could be more subtle than that, and for example Alice might
  need to \emph{prove} to the CA that she does indeed know the
  corresponding secret key.} Now Alice can send
\((v_{Alice},\sigma_{Alice})\) to Bob to certify that the owner of this
public key is indeed Alice.

For example, in the web setting, certain
\href{https://en.wikipedia.org/wiki/Certificate_authority}{certificate
authorities} can certify that a certain public key is associated with a
certain website. If you go to a website using the \texttt{https}
protocol, you should see a ``lock'' symbol on your browser which will
give you details on the certificate. Often the certificate is a chain of
certificate. If I click on this lock symbol in my Chrome browser, I see
that the certificate that amazon.com's public key is some particular
string (corresponding to a 2048 RSA modulos and exponent) is signed by
the Symantec Certificate authority, whose own key is certified by
Verisign. My communication with Amazon is an example of a setting of
\emph{one sided authentication}. It is important for me to know that I
am truly talking to amazon.com, while Amazon is willing to talk to any
client. (Though of course once we establish a secure channel, I could
use it to login to my Amazon account.) Chapter 21 of Boneh Shoup
contains an in depth discussion of authenticated key exchange protocols.

\begin{pause} \label[pause]{14-You-should-stop-here-a}

You should stop here and read Section 21.9 of Boneh Shoup with the
formal definitions of authenticated key exchange, going back as needed
to the previous section for the definitions of protocols AEK1 - AEK4.

\end{pause}

\subsection{The compiler approach for authenticated key
exchange}\label{14-The-compiler-approach-}

There is a generic ``compiler'' approach to obtaining authenticated key
exchange protocols:

\begin{itemize}
\item
  Start with a protocol such as the basic Diffie-Hellman protocol that
  is only secure with respect to a \emph{passive eavesdropping}
  adversary.
\item
  Then \emph{compile} it into a protocol that is secure with respect to
  an active adversary using authentication tools such as digital
  signatures, message authentication codes, etc.., depending on what
  kind of setup you can assume and what properties you want to achieve.
\end{itemize}

This approach has the advantage of being modular in both the
construction and the analysis. However, direct constructions might be
more efficient. There are a great many potentially desirable properties
of key exchange protocols, and different protocols achieve different
subsets of these properties at different costs. The most common variant
of authenticated key exchange protocols is to use some version of the
Diffie-Hellman key exchange. If both parties have public signature keys,
then they can simply sign their messages and then that effectively rules
out an active attack, reducing active security to passive security
(though one needs to include identities in the signatures to ensure non
repeating of messages, see
\href{http://link.springer.com/article/10.1007\%2FBF00124891}{here}).

The most efficient variants of Diffie Hellman achieve authentication
implicitly, where the basic protocol remains the same (sending \(X=g^x\)
and \(Y=g^y\)) but the computation of the secret shared key involves
some authentication information. Of these protocols a particularly
efficient variant is the MQV protocol of Law, Menezes, Qu, Solinas and
Vanstone (which is based on similar principles as DSA signatures), and
its variant \href{https://eprint.iacr.org/2005/176.pdf}{HMQV} by
Krawczyk that has some improved security properties and analysis.

\section{Password authenticated key
exchange.}\label{14-Password-authenticated}

To be completed (the most natural candidate: use MACS with a
password-derived key to authenticate communication - completely fails)

\begin{pause} \label[pause]{14-Please-skim-Boneh-Shou}

Please skim Boneh Shoup Chapter 21.11

\end{pause}

\section{Client to client key exchange for secure text messaging - ZRTP,
OTR, TextSecure}\label{14-Client-to-client-key-e}

To be completed. See
\href{http://blog.cryptographyengineering.com/2013/03/here-come-encryption-apps.html}{Matthew
Green's blog} ,
\href{https://whispersystems.org/blog/advanced-ratcheting/}{text
secure}, \href{https://otr.cypherpunks.ca/Protocol-v3-4.0.0.html}{OTR}.

Security requirements: forward secrecy, deniability.

\section{Heartbleed and logjam attacks}\label{14-Heartbleed-and-logjam-}

\begin{itemize}
\item
  Vestiges of past crypto policies.
\item
  Importance of ``perfect forward secrecy''
\end{itemize}

\begin{marginfigure}
\centering
\includegraphics[width=\linewidth, height=1.5in, keepaspectratio]{../figure/NSA_Page_29.jpg}
\caption{How the NSA feels about breaking encrypted communication}
\label{tmplabelfig}
\end{marginfigure}
