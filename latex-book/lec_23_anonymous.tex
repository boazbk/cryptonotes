\chapter{Anonymous communication}\label{Anonymous-communication}

Encryption is meant to protect the contents of communication, but
sometimes the bigger secret is that the communication existed in the
first place. If a whistleblower wants to leak some information to the
New York Times, the mere fact that she sent an email would reveal her
identity. There are two main concepts aimed at achieving anonymity:

\begin{itemize}
\item
  \emph{Anonymous routing} is about ensuring that Alice and Bob can
  communicate without that fact being revealed.
\item
  \emph{Steganography} is about having Alice and Bob hiding an encrypted
  communication in the context of an seemingly innocuous conversation.
\end{itemize}

\section{Steganography}\label{Steganography}

The goal in a stegnaographic communication is to hide cryptographic (or
non cryptographic) content without being detected. The idea is simple:
let's start with the \emph{symmetric case} and assume Alice and Bob
share a shared key \(k\) and Alice wants to transmit a bit \(b\) to Bob.
We assume that Alice and has a choice of \(t\) words \(w_1,\ldots,w_t\)
that would be reasonable for her to send at this point in the
conversation. Alice will choose a word \(w_i\) such that \(f_k(w_i)=b\)
where \(\{ f_k \}\) is a pseudorandom function collection. With
probability \(1-2^{-t}\) there will be such a word. Bob will decode the
message using \(f_k(w_i)\). Alice and Bob can use an error correcting
code to compensate for the probability \(2^{-t}\) that Alice is forced
to send the wrong bit.

In the \emph{public key setting}, suppose that Bob publishes a public
key \(e\) for an encryption scheme that has \emph{pseudorandom
ciphertexts}. That is, to a party that does not know the key, an
encryption is indistinguishable from a random string. To send some
message \(m\) to Bob, Alice computes \(c = E_e(m)\) and transmits it to
Bob one bit at a time. Given the \(t\) words \(w_1,\ldots,w_t\), to
transmit the bit \(c_j\) Alice chooses a word \(w_i\) such that
\(H(w_i)=c_j\) where \(H:\{0,1\}^*\rightarrow\{0,1\}\) is a hash
function modeled as a random oracle. The distribution of words output by
Alice \(w^1,\ldots,w^\ell\) is uniform conditioned on
\((H(w^1),\ldots,H(w^\ell))=c\). But note that if \(H\) is a random
oracle, then \(H(w^1),\ldots,H(w^\ell)\) is going to be uniform, and
hence indistinguishable from \(c\).

\section{Anonymous routing}\label{Anonymous-routing}

\begin{itemize}
\item
  \textbf{Low latency communication:} Aqua, Crowds, LAP, ShadowWalker,
  Tarzan, Tor
\item
  \textbf{Message at a time, protection against timing / traffic
  analysis:} Mix-nets, e-voting, Dining Cryptographer network (DC net),
  Dissent, Herbivore, Riposte
\end{itemize}

\section{Tor}\label{Tor}

Basic arhictecture. Attacks

\section{Telex}\label{Telex}

\section{Riposte}\label{Riposte}
