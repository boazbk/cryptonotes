\chapter{Preface}\label{prefacechap}

\begin{quote}
\emph{``We make ourselves no promises, but we cherish the hope that the
unobstructed pursuit of useless knowledge will prove to have
consequences in the future as in the past''} \ldots{} \emph{``An
institution which sets free successive generations of human souls is
amply justified whether or not this graduate or that makes a so-called
useful contribution to human knowledge. A poem, a symphony, a painting,
a mathematical truth, a new scientific fact, all bear in themselves all
the justification that universities, colleges, and institutes of
research need or require''}, Abraham Flexner,
\href{https://library.ias.edu/files/UsefulnessHarpers.pdf}{The
Usefulness of Useless Knowledge}, 1939.
\end{quote}

\begin{quote}
\emph{``I suggest that you take the hardest courses that you can,
because you learn the most when you challenge yourself\ldots{} CS 121 I
found pretty hard.''}, \href{https://youtu.be/xFFs9UgOAlE?t=3646}{Mark
Zuckerberg}, 2005.
\end{quote}

This is a textbook for an undergraduate introductory course on
Theoretical Computer Science. The educational goals of this book are to
convey the following:

\begin{itemize}
\item
  That computation arises in a variety of natural and human-made
  systems, and not only in modern silicon-based computers.
\item
  Similarly, beyond being an extremely important \emph{tool},
  computation also serves as a useful \emph{lens} to describe natural,
  physical, mathematical and even social concepts.
\item
  The notion of \emph{universality} of many different computational
  models, and the related notion of the duality between \emph{code} and
  \emph{data}.
\item
  The idea that one can precisely define a mathematical model of
  computation, and then use that to prove (or sometimes only conjecture)
  lower bounds and impossibility results.
\item
  Some of the surprising results and discoveries in modern theoretical
  computer science, including the prevalence of NP-completeness, the
  power of interaction, the power of randomness on one hand and the
  possibility of derandomization on the other, the ability to use
  hardness ``for good'' in cryptography, and the fascinating possibility
  of quantum computing.
\end{itemize}

I hope that following this course, students would be able to recognize
computation, with both its power and pitfalls, as it arises in various
settings, including seemingly ``static'' content or ``restricted''
formalisms such as macros and scripts. They should be able to follow
through the logic of \emph{proofs} about computation, including the
central concept of a \emph{reduction}, as well as understanding
``self-referential'' proofs (such as diagonalization-based proofs that
involve programs given their own code as input). Students should
understand that some problems are \emph{inherently intractable}, and be
able to recognize the potential for intractability when they are faced
with a new problem. While this book only touches on cryptography,
students should understand the basic idea of how we can use
computational hardness for cryptographic purposes. However, more than
any specific skill, this book aims to introduce students to a new way of
thinking of computation as an object in its own right and to illustrate
how this new way of thinking leads to far-reaching insights and
applications.

My aim in writing this text is to try to convey these concepts in the
simplest possible way and try to make sure that the formal notation and
model help elucidate, rather than obscure, the main ideas. I also tried
to take advantage of modern students' familiarity (or at least
interest!) in programming, and hence use (highly simplified) programming
languages to describe our models of computation. That said, this book
does not assume fluency with any particular programming language, but
rather only some familiarity with the general \emph{notion} of
programming. We will use programming metaphors and idioms, occasionally
mentioning specific programming languages such as \emph{Python},
\emph{C}, or \emph{Lisp}, but students should be able to follow these
descriptions even if they are not familiar with these languages.

Proofs in this book, including the existence of a universal Turing
Machine, the fact that every finite function can be computed by some
circuit, the Cook-Levin theorem, and many others, are often constructive
and algorithmic, in the sense that they ultimately involve transforming
one program to another. While it is possible to follow these proofs
without seeing the code, I do think that having access to the code, and
the ability to play around with it and see how it acts on various
programs, can make these theorems more concrete for the students. To
that end, an accompanying website (which is still work in progress)
allows executing programs in the various computational models we define,
as well as see constructive proofs of some of the theorems.

\section{To the student}\label{To-the-student}

This book can be challenging, mainly because it brings together a
variety of ideas and techniques in the study of computation. There are
quite a few technical hurdles to master, whether it is following the
diagonalization argument for proving the Halting Problem is undecidable,
combinatorial gadgets in NP-completeness reductions, analyzing
probabilistic algorithms, or arguing about the adversary to prove the
security of cryptographic primitives.

The best way to engage with this material is to read these notes
\textbf{actively}, so make sure you have a pen ready. While reading, I
encourage you to stop and think about the following:

\begin{itemize}
\item
  When I state a theorem, stop and take a shot at proving it on your own
  \emph{before} reading the proof. You will be amazed by how much better
  you can understand a proof even after only 5 minutes of attempting it
  on your own.
\item
  When reading a definition, make sure that you understand what the
  definition means, and what the natural examples are of objects that
  satisfy it and objects that do not. Try to think of the motivation
  behind the definition, and whether there are other natural ways to
  formalize the same concept.
\item
  Actively notice which questions arise in your mind as you read the
  text, and whether or not they are answered in the text.
\end{itemize}

As a general rule, it is more important that you understand the
\textbf{definitions} than the \textbf{theorems}, and it is more
important that you understand a \textbf{theorem statement} than its
\textbf{proof}. After all, before you can prove a theorem, you need to
understand what it states, and to understand what a theorem is about,
you need to know the definitions of the objects involved. Whenever a
proof of a theorem is at least somewhat complicated, I provide a ``proof
idea.'' Feel free to skip the actual proof in a first reading, focusing
only on the proof idea.

This book contains some code snippets, but this is by no means a
programming text. You don't need to know how to program to follow this
material. The reason we use code is that it is a \emph{precise} way to
describe computation. Particular implementation details are not as
important to us, and so we will emphasize code readability at the
expense of considerations such as error handling, encapsulation, etc.
that can be extremely important for real-world programming.

\subsection{Is the effort worth it?}\label{Is-the-effort-worth-it}

This is not an easy book, and you might reasonably wonder why should you
spend the effort in learning this material. A traditional justification
for a ``Theory of Computation'' course is that you might encounter these
concepts later on in your career. Perhaps you will come across a hard
problem and realize it is NP complete, or find a need to use what you
learned about regular expressions. This might very well be true, but the
main benefit of this book is not in teaching you any practical tool or
technique, but instead in giving you a \emph{different way of thinking}:
an ability to recognize computational phenomena even when they occur in
non-obvious settings, a way to model computational tasks and questions,
and to reason about them.

Regardless of any use you will derive from this book. I believe learning
this material is important because it contains concepts that are both
beautiful and fundamental. The role that \emph{energy} and \emph{matter}
played in the 20th century is played in the 21st by \emph{computation}
and \emph{information}, not just as tools for our technology and
economy, but also as the basic building blocks we use to understand the
world. This book will give you a taste of some of the theory behind
those, and hopefully spark your curiosity to study more.

\section{To potential instructors}\label{To-potential-instructors}

I wrote this book for my Harvard course, but I hope that other lecturers
will find it useful as well. To some extent, it is similar in content to
``Theory of Computation'' or ``Great Ideas'' courses such as those
taught at \href{http://www.cs.cmu.edu/~./15251/}{CMU} or
\href{http://stellar.mit.edu/S/course/6/sp16/6.045/materials.html}{MIT}.

The most significant difference between our approach and more
traditional ones (such as Hopcroft and Ullman's
\cite{HopcroftUllman69, HopcroftUllman79} and Sipser's
\cite{SipserBook}) is that we do not start with \emph{finite automata}
as our initial computational model. Instead, our initial computational
model is \emph{Boolean Circuits}.\footnote{An earlier book that starts
  with circuits as the initial model is John Savage's
  \cite{Savage1998models}.} We believe that Boolean Circuits are more
fundamental to the theory of computing (and even its practice!) than
automata. In particular, Boolean Circuits are a prerequisite for many
concepts that one would want to teach in a modern course on Theoretical
Computer Science, including cryptography, quantum computing,
derandomization, attempts at proving \(\mathbf{P} \neq \mathbf{NP}\),
and more. Even in cases where Boolean Circuits are not strictly
required, they can often offer significant simplifications (as in the
case of the proof of the Cook-Levin Theorem).

Furthermore, I believe there are pedagogical reasons to start with
Boolean circuits as opposed to finite automata. Boolean circuits are a
more natural model of computation, and one that corresponds more closely
to computing in Silicon, making the connection to practice more
immediate to the students. Finite functions are arguably easier to grasp
than infinite ones, as we can fully write down their truth table. The
theorem that \emph{every} finite function can be computed by some
Boolean circuit is both simple enough and important enough to serve as
an excellent starting point for this course. Moreover, many of the main
conceptual points of the theory of computation, including the notions of
the duality between \emph{code} and \emph{data}, and the idea of
\emph{universality}, can already be seen in this context.

After Boolean circuits, we move on to Turing machines and prove results
such as the existence of a universal Turing machine, the uncomputability
of the halting problem, and Rice's Theorem. Automata are discussed after
we see Turing machines and undecidability, as an example for a
\emph{restricted computational model} where problems such as determining
halting can be effectively solved.

While this is not our motivation, the order we present circuits, Turing
machines, and automata roughly corresponds to the chronological order of
their discovery. Boolean algebra goes back to Boole's and DeMorgan's
works in the 1840s \cite{Boole1847mathematical, DeMorgan1847} (though
the definition of Boolean circuits and the connection to physical
computation was given 90 years later by Shannon \cite{Shannon1938}).
Alan Turing defined what we now call ``Turing Machines'' in the 1930s
\cite{Turing37}, while finite automata were introduced in the 1943 work
of McCulloch and Pitts \cite{McCullochPitts43} but only really
understood in the seminal 1959 work of Rabin and Scott
\cite{RabinScott59}.

More importantly, while models such as finite-state machines, regular
expressions, and context-free grammars are incredibly important for
practice, the main applications for these models (whether it is for
parsing, for analyzing properties such as \emph{liveness} and
\emph{safety}, or even for
\href{https://www.cs.cornell.edu/~kozen/Papers/NetKAT-APLAS.pdf}{software-defined
routing tables}) rely crucially on the fact that these are
\emph{tractable} models for which we can effectively answer
\emph{semantic questions}. This practical motivation can be better
appreciated \emph{after} students see the undecidability of semantic
properties of general computing models.

The fact that we start with circuits makes proving the Cook-Levin
Theorem much easier. In fact, our proof of this theorem can (and is)
done using a handful of lines of Python. Combining this proof with the
standard reductions (which are also implemented in Python) allows
students to appreciate visually how a question about computation can be
mapped into a question about (for example) the existence of an
independent set in a graph.

Some other differences between this book and previous texts are the
following:

\begin{enumerate}
\def\labelenumi{\arabic{enumi}.}
\item
  For measuring \emph{time complexity}, we use the standard RAM machine
  model used (implicitly) in algorithms courses, rather than Turing
  machines. While these two models are of course polynomially
  equivalent, and hence make no difference for the definitions of the
  classes \(\mathbf{P}\), \(\mathbf{NP}\), and \(\mathbf{EXP}\), our
  choice makes the distinction between notions such as \(O(n)\) or
  \(O(n^2)\) time more meaningful. This choice also ensures that these
  finer-grained time complexity classes correspond to the informal
  definitions of linear and quadratic time that students encounter in
  their algorithms lectures (or their whiteboard coding
  interviews\ldots).
\item
  We use the terminology of \emph{functions} rather than
  \emph{languages}. That is, rather than saying that a Turing Machine
  \(M\) \emph{decides a language} \(L \subseteq \{0,1\}^*\), we say that
  it \emph{computes a function} \(F:\{0,1\}^* \rightarrow \{0,1\}\). The
  terminology of ``languages'' arises from Chomsky's work
  \cite{Chomsky56}, but it is often more confusing than illuminating.
  The language terminology also makes it cumbersome to discuss concepts
  such as algorithms that compute functions with more than one bit of
  output (including basic tasks such as addition, multiplication,
  etc\ldots). The fact that we use functions rather than languages means
  we have to be extra vigilant about students distinguishing between the
  \emph{specification} of a computational task (e.g., the
  \emph{function}) and its \emph{implementation} (e.g., the
  \emph{program}). On the other hand, this point is so important that it
  is worth repeatedly emphasizing and drilling into the students,
  regardless of the notation used. The book does mention the language
  terminology and reminds of it occasionally, to make it easier for
  students to consult outside resources.
\end{enumerate}

Reducing the time dedicated to finite automata and context-free
languages allows instructors to spend more time on topics that a modern
course in the theory of computing needs to touch upon. These include
randomness and computation, the interactions between \emph{proofs} and
\emph{programs} (including Gödel's incompleteness theorem, interactive
proof systems, and even a bit on the \(\lambda\)-calculus and the
Curry-Howard correspondence), cryptography, and quantum computing.

This book contains sufficient detail to enable its use for self-study.
Toward that end, every chapter starts with a list of learning
objectives, ends with a recap, and is peppered with ``pause boxes''
which encourage students to stop and work out an argument or make sure
they understand a definition before continuing further.

\cref{roadmapsec} contains a ``roadmap'' for this book, with
descriptions of the different chapters, as well as the dependency
structure between them. This can help in planning a course based on this
book.

\section{Acknowledgements}\label{Acknowledgements}

This text is continually evolving, and I am getting input from many
people, for which I am deeply grateful. Thanks to Scott Aaronson,
Michele Amoretti, Marguerite Basta, Sam Benkelman, Jarosław Błasiok,
Emily Chan Christy Cheng, Michelle Chiang, Daniel Chiu, Chi-Ning Chou,
Michael Colavita, Robert Darley Waddilove, Juan Esteller, David Evans,
Leor Fishman, William Fu, Thomas Huet, Piotr Galuszka, Mark Goldstein,
Alexander Golovnev, Chan Kang, Nina Katz-Christy, Eddie Kohler,
Estefania Lahera, Allison Lee, Ondřej Lengál, Raymond Lin, Emma Ling,
Alex Lombardi, Lisa Lu, Aditya Mahadevan, Jacob Meyerson, George Moe,
Hamish Nicholson, Sandip Nirmel, Sebastian Oberhoff, Thomas Orton, Pablo
Parrilo, Juan Perdomo, Aaron Sachs, Abdelrhman Saleh, Brian Sapozhnikov,
Peter Schäfer, Anthony Scemama, Josh Seides, Alaisha Sharma, Noah
Singer, Matthew Smedberg, Hikari Sorensen, Alec Sun, Amol Surati,
Everett Sussman, Marika Swanberg, Garrett Tanzer, Sarah Turnill, Salil
Vadhan, Patrick Watts, Ryan Williams, Licheng Xu, Wanqian Yang,
Elizabeth Yeoh-Wang, Josh Zelinsky, Grace Zhang, and Jessica Zhu for
helpful feedback.

If you have any comments or suggestions, please do post them on the
GitHub repository \url{https://github.com/boazbk/tcs}.

Salil Vadhan co-taught with me the first iteration of this course and
gave me a tremendous amount of useful feedback and insights during this
process. Michele Amoretti and Marika Swanberg carefully read several
chapters of this text and gave extremely helpful detailed comments. Dave
Evans contributed many pull requests fixing errors and improving
phrasing.

Thanks to Anil Ada, Venkat Guruswami, and Ryan O'Donnell for helpful
tips from their experience in teaching
\href{http://www.cs.cmu.edu/~./15251/}{CMU 15-251}. Juan Esteller and
Gabe Montague initially implemented the NAND* programming languages in
OCaml and Javascript.

I am using many open source software packages in the production of these
notes for which I am grateful. In particular, I am thankful to Donald
Knuth and Leslie Lamport for
\href{https://www.latex-project.org/}{LaTeX} and to John MacFarlane for
\href{http://pandoc.org/}{Pandoc}. David Steurer wrote the original
scripts to produce this text. The current version uses Sergio Correia's
\href{http://scorreia.com/software/panflute/}{panflute}. The templates
for the LaTeX and HTML versions are derived from
\href{https://tufte-latex.github.io/tufte-latex/}{Tufte LaTeX},
\href{https://www.gitbook.com/}{Gitbook} and
\href{https://bookdown.org/}{Bookdown}. Thanks to Amy Hendrickson for
some LaTeX consulting. I used the \href{http://jupyter.org/}{Jupyter
project} to write the supplemental code snippets.

Finally, I would like to thank my family: my wife Ravit, and my children
Alma and Goren. Working on this book (and the corresponding course) took
so much of my time that Alma wrote an essay for her fifth-grade class
saying that ``universities should not pressure professors to work too
much.'' I'm afraid all I have to show for this effort is 600 pages of
ultra-boring mathematical text.
